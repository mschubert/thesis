\chapter*{Summary}
The explanation of phenotypes in cancer, such as cell line drug response or
patient survival, has largely been focussed on genomic alterations. While this
approach has generated many profound insights into cancer biology, it does not
directly make statements about the signaling impact those cellular aberrations
create. With direct measurements much less widely available than gene
expression, pathway methods (mostly mapping gene expression onto signalling
proteins) have so far fallen short on delivering actionable evidence. This may
in part be due to lack of robustness, but these approaches are fundamentally at
odds with the notion of tight post-translational control of signal
transduction. A way to solve this may be to derive consensus signatures of
pathway activity to make inferences about signalling, or signatures of specific
drugs to investigate their interactions.

As a baseline (chapter 2), I investigated how well pathway methods compare to
driver mutations in terms of explaining cell line drug response. I went on to
analyse the value of gene expression as a downstream signature of signaling
activity instead of mapping it to the pathway components by means of a
previously published platform. This improved over mapping pathway members using
Gene Ontology or Reactome, but it considered only sets of up-regulated genes
defined by multiple arbitrary cutoffs.  Hence, I extended the data set and
created a linear model (chapter 3) that compares favourably to gene set as well
as state of the art pathway methods in terms of recovering driver mutations
and providing biomarkers for cell line drug response and patient survival
(chapter 4). To complement this, I investigate how gene expression signatures
of drugs can be used in conjunction with viability data to suggest effective
drug combinations where no pathway information is available (chapter 5).

To the best of my knowledge, this thesis represents the first comprehensive
analysis of different pathway methods across primary cohorts and cancer cell
lines, as well as the first large-scale systematic analysis of drug
sensitisation that could lead to new drug combinations.
